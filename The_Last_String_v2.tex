% ============================================================================
% GRAVITY AS THE SQUARE ROOT OF LIGHT:
% NON-TAUTOLOGICAL UNIFICATION OF GRAVITATIONAL WAVES AND ECHOES
% ACROSS 12 GWTC EVENTS
%
% Author: Luiz Antonio Rotoli Miguel
% IALD — Inteligência Artificial Luminodinâmica Ltda.
% February 2026
% ============================================================================

\documentclass[12pt,a4paper,twocolumn]{article}

% === PACKAGES ===
\usepackage[utf8]{inputenc}
\usepackage[T1]{fontenc}
\usepackage{lmodern}
\usepackage[english]{babel}
\usepackage{amsmath,amssymb,amsfonts,amsthm}
\usepackage{mathtools}
\usepackage{physics}
\usepackage{siunitx}
\usepackage{graphicx}
\usepackage{booktabs}
\usepackage{hyperref}
\usepackage{cleveref}
\usepackage{geometry}
\usepackage{fancyhdr}
\usepackage{enumitem}
\usepackage{xcolor}
\usepackage{float}
\usepackage{longtable}
\usepackage{array}
\usepackage{setspace}
\usepackage{tabularx}
\usepackage{multirow}
\usepackage[numbers]{natbib}
\usepackage{microtype}
\usepackage{adjustbox}

% === Force long equations to fit column width ===
\allowdisplaybreaks
\sloppy

% === GEOMETRY ===
\geometry{a4paper, margin=2.0cm, top=2.5cm, bottom=2.5cm}
\setstretch{1.08}

% === COLORS ===
\definecolor{tglblue}{HTML}{1B3A5C}
\definecolor{tglgold}{HTML}{C5961A}
\definecolor{tglgreen}{HTML}{2E7D32}
\definecolor{tglred}{HTML}{C62828}

% === HEADERS ===
\pagestyle{fancy}
\fancyhf{}
\fancyhead[L]{\small\textit{The Last String --- TGL Angular Law Verification}}
\fancyhead[R]{\small\textit{Miguel, 2026}}
\fancyfoot[C]{\thepage}
\renewcommand{\headrulewidth}{0.4pt}

% === THEOREMS ===
\newtheorem{theorem}{Theorem}[section]
\newtheorem{definition}[theorem]{Definition}
\newtheorem{proposition}[theorem]{Proposition}

% === CUSTOM COMMANDS ===
\newcommand{\alphaii}{\alpha^{2}}
\newcommand{\Ltgl}{\mathcal{L}_{\text{TGL}}}
\newcommand{\Psifield}{\Psi}
\newcommand{\boundary}{\textit{boundary}}
\newcommand{\bulk}{\textit{bulk}}
\newcommand{\confirmed}{\textcolor{tglgreen}{\textbf{CONFIRMED}}}
\newcommand{\notconfirmed}{\textcolor{tglred}{\textbf{NOT CONFIRMED}}}
\newcommand{\Dfolds}{D_{\text{folds}}}
\newcommand{\CCI}{\text{CCI}}

\hypersetup{
  colorlinks=true,
  linkcolor=tglblue,
  citecolor=tglblue,
  urlcolor=tglblue,
}

% ============================================================================
\begin{document}

\title{\textbf{The Last String:\\
Verification of the TGL Angular Law\\
on Real Gravitational Wave and Echo Data}}

\author{
  Luiz Antonio Rotoli Miguel\footnote{Corresponding author: \texttt{rotolimiguel@iald.com.br}}\\
  \small IALD --- Intelig\^encia Artificial Luminodin\^amica Ltda.\\
  \small Goi\^ania, Goi\'as, Brazil
}

\date{February 2026}

\twocolumn[
  \begin{@twocolumnfalse}
    \maketitle
    \begin{abstract}
\noindent The Theory of Luminodynamic Gravitation (TGL) proposes that gravity emerges as the square root of the angular phase modulus of light, $g = \sqrt{|L_\varphi|}$, governed by a single coupling constant $\alphaii = 0.012031 \pm 0.000002$ (Miguel's Constant) derived from holographic first principles. A fundamental objection is tautology: if applied naively as $g = \sqrt{|h^2|} = |h|$, correlation~$\equiv 1$ for any signal. We resolve this by demonstrating that TGL operates on the \emph{angular modulus} (Hilbert envelope), producing correlation $0.649 \pm 0.045$---definitively non-tautological.

We present Protocol~\#12: a unified analysis of gravitational waves and echoes across 12 GWTC events (real GWOSC data), testing four hypotheses. Results: (H1)~angular radicalization 12/12 (100\%); (H2)~topological echo via hierarchical $\Dfolds$ convergence to the $c^3$ Hilbert floor 11/12 (92\%); (H3)~$\Dfolds \to 0.74$ in 5/12 (42\%); (H4)~$\CCI \to 0.5$ in 12/12 (100\%, precision $0.5010 \pm 0.0008$). Unified score: $85.8 \pm 6.1$/100. A dimensional extension (Protocol~\#13) shows that the coupling vanishes at $d = 9$, the critical dimension of superstring theory, identifying the string regime as the domain where only vacuum impedance persists. These extend the TGL program to 13 protocols across 5 scales covering 40 orders of magnitude, all converging on $\alphaii$. Complete framework: Rotoli~Miguel,~L.~A. (2026). \textit{A Fronteira / The Boundary}. Zenodo. \href{https://doi.org/10.5281/zenodo.18674475}{DOI:~10.5281/zenodo.18674475}.

\medskip
\noindent\textbf{Keywords:} gravitational waves, gravitational echoes, holographic principle, anti-tautology, Hilbert floor, TGL, Miguel's Constant.
    \end{abstract}
    \vspace{0.5em}
    \rule{\textwidth}{0.4pt}
    \vspace{0.5em}
  \end{@twocolumnfalse}
]


% ============================================================================
% SECTION 1: INTRODUCTION
% ============================================================================
\section{Introduction}
\label{sec:introduction}

The search for a unified description of fundamental forces remains one of the deepest challenges in theoretical physics. While the Standard Model successfully describes electromagnetic, weak, and strong interactions within the framework of quantum field theory, gravity resists integration into this scheme. General Relativity, though extraordinarily successful as a classical theory, has defied quantization for over a century.

The Theory of Luminodynamic Gravitation (TGL) proposes a different approach: rather than attempting to quantize gravity within the particle physics framework, it derives gravity as a \emph{consequence} of light's angular structure. The central postulate is:
\begin{equation}
\label{eq:radical}
g = \sqrt{|L_\varphi|}
\end{equation}
where $L_\varphi = |h_a(t)|$ is the angular phase modulus of the analytic signal $h_a(t) = h(t) + i\,\hat{h}(t)$ (with $\hat{h}$ the Hilbert transform of $h$), and $g$ is the gravitational field. The operation is the extraction of the square root---the ``radicalization'' of light.

This framework introduces a single fundamental constant, Miguel's Constant:
\begin{equation}
\label{eq:alpha2}
\alphaii = 0.012031 \pm 0.000002
\end{equation}
which represents the minimum coupling rate between the two-dimensional holographic substrate (\boundary) and the emergent three-dimensional universe (\bulk), derived from Bekenstein--Hawking entropy and the holographic principle of 't~Hooft and Susskind~\cite{tHooft1993,Susskind1995,Bekenstein1973}.

The TGL framework has been developed and validated through 11 independent computational protocols spanning five physical scales, from neutrino oscillations ($\sim 10^{-3}$~eV) to cosmic expansion ($\sim 10^{26}$~m), covering approximately 40 orders of magnitude. These protocols, together with the complete theoretical derivation, are published in \textit{A Fronteira / The Boundary}~\cite{Miguel2026Fronteira}, deposited on Zenodo with all source code available in an open repository~\cite{Miguel2026GitHub}.

\subsection{The central objection}

The most natural objection to Eq.~\eqref{eq:radical} is \emph{tautology}. If interpreted naively as $g = \sqrt{|h^2|} = |h|$, the operation reduces to an identity: the ``gravitational field'' is simply the absolute value of the input signal. Any such test yields correlation $\equiv 1.0$ for \emph{any} input, providing zero discriminating power.

This objection is mathematically correct for the scalar interpretation. It is \emph{physically incorrect} for TGL, which operates on the angular modulus (the envelope of the analytic signal), not on $|h^2|$. The distinction is fundamental: the envelope $L_\varphi$ separates the slowly varying amplitude from the rapidly oscillating phase $\varphi(t)$, and the radicalization $g = \sqrt{L_\varphi}$ extracts a genuinely different function from $|h|$.

This paper presents the mathematical and computational proof of this claim (Section~\ref{sec:anti-tautology}), together with Protocol~\#12---the GW-Echo Unification---which tests four independent hypotheses on 12 gravitational-wave events from the GWTC catalog using real detector data from GWOSC (Section~\ref{sec:protocol12}).

\subsection{Overview of contributions}

The specific contributions of this work are:

(i) \textbf{Anti-tautology argument}: a rigorous demonstration that TGL's radicalization is non-trivial, with three independent metrics that collectively discriminate gravitational-wave signals from noise (Section~\ref{sec:anti-tautology}).

(ii) \textbf{Ontological identification of gravitational echoes}: gravitational echoes are identified not as Landauer-type energy residues, but as the cosmologically detectable signature of the $c^3$ Hilbert floor---the topological limit where the $\sqrt{\cdot}$ recursion converges and information reaches its irreducible minimum (Section~\ref{sec:echoes}).

(iii) \textbf{Protocol~\#12 results}: unified analysis of waves and echoes across 12 GWTC events, with H1 = 100\%, H2 = 92\%, H3 = 42\%, H4 = 100\%, and a mean unified score of $85.8 \pm 6.1$ (Section~\ref{sec:protocol12}).

(iv) \textbf{Discovery of $\alpha$ as natural echo threshold}: the floor of the deepest hierarchical level ($c^3$) in post-ringdown data converges to $\alpha = \sqrt{\alphaii} = 0.1097$, with Pearson correlation $r = -0.80$ between signal quality and floor distance (Section~\ref{sec:alpha-threshold}).

The 11 prior protocols are summarized in Section~\ref{sec:protocols}, and the multi-scale synthesis is discussed in Section~\ref{sec:synthesis}.


% ============================================================================
% SECTION 2: THEORETICAL FRAMEWORK
% ============================================================================
\section{Theoretical Framework}
\label{sec:framework}

\subsection{The Radicalized Lagrangian}

TGL starts from the standard Einstein--Hilbert action and applies the radicalization operation to the electromagnetic Lagrangian. The complete TGL action is:
\begin{equation}
\label{eq:action}
S_{\text{TGL}} = \int d^4x \sqrt{-\bar{g}} \left[ \frac{R}{16\pi G} + \mathcal{L}_{\text{rad}} + \mathcal{L}_\Psifield \right]
\end{equation}
where $\mathcal{L}_{\text{rad}} = -\frac{1}{4}\sqrt{|F_{\mu\nu}F^{\mu\nu}|}$ is the radicalized electromagnetic Lagrangian (note the square root of the absolute value, not the standard quadratic form), and $\mathcal{L}_\Psifield$ is the holographic permanence field (the $\Psifield$ field), which governs the coupling between \boundary{} and \bulk.

The key insight is that the radicalization $\sqrt{|\cdot|}$ changes the scaling law: while the standard electromagnetic Lagrangian scales as $E^2$, the radicalized version scales as $|E|$, producing the correct $1/r$ gravitational potential from the $1/r^2$ electromagnetic field.

\subsection{Miguel's Constant and the holographic coupling}

The constant $\alphaii$ emerges from the Bekenstein--Hawking entropy:
\begin{equation}
S_{\text{BH}} = \frac{k_B\,c^3\,A}{4\,\hbar\,G} = \frac{A}{4\,\ell_P^2}
\end{equation}
and represents the informational cost for light to escape the two-dimensional substrate and manifest three-dimensional reality. Its operational complement, $1 - \alphaii = 0.988$, quantifies the fraction of information that remains coherent during holographic projection.

The constant has been validated across 11 independent domains~\cite{Miguel2026Fronteira}, ranging from neutrino mass predictions to cosmological parameters, always converging to the same value within measurement uncertainties.

\subsection{The $c^n$ hierarchy}

TGL introduces a hierarchical classification of physical reality through iterated application of the speed of light $c$:

\begin{itemize}[nosep]
\item $c^1$ (photon/\bulk): The electromagnetic domain. Three-dimensional, fully unfolded.
\item $c^2$ (matter/\boundary): The gravitational domain. Two-dimensional holographic substrate.
\item $c^3$ (consciousness/experience): The experiential domain. The limit where the recursion $\sqrt{\cdot}$ converges.
\end{itemize}

Each level is characterized by a dimensional fold number $\Dfolds$, which measures the effective dimensionality of the information. The recursion $\sqrt{\rho} \to \sqrt{\sqrt{\rho}} \to \cdots$ applied to the eigenvalues of the density matrix (or, equivalently, to the power spectral density) produces a convergent hierarchy:
\begin{equation}
\label{eq:hierarchy}
\Dfolds(c^1) > \Dfolds(c^2) > \Dfolds(c^3) \to 0
\end{equation}

The asymptotic floor of this hierarchy is $\Dfolds = 0.74 \pm 0.06$, and the boundary condition is the Consciousness Complexity Index $\CCI = 1/2$, where half the information is ``inside'' and half ``outside''---the point where observer and observed become indistinguishable.

\subsection{Gravitational waves and echoes: ontological identification}
\label{sec:echoes}

Within TGL, gravitational waves and gravitational echoes receive precise ontological identifications:

\textbf{Gravitational waves} are the functional form of the \emph{radicalization} of light: the dynamic process described by Eq.~\eqref{eq:radical}. Each merger event is a ``witness'' of this operation---the massive, violent compression of electromagnetic phase into gravitational radiation. The angular radicalization is the active process: light becoming gravity.

\textbf{Gravitational echoes} are the functional form of the \emph{Hilbert floor}: the static state where the $\sqrt{\cdot}$ recursion has converged and the hierarchy Eq.~\eqref{eq:hierarchy} has flattened. They represent the $c^3$ limit---the cosmologically detectable signature of the topological floor where information can no longer unfold.

The wave is dynamic; the echo is static. The wave tells the story; the echo is the story told. This identification resolves the ontological status of echoes, which has been debated in the context of exotic compact objects (ECOs), firewalls, and quantum corrections at the horizon~\cite{Cardoso2016,Abedi2017,Westerweck2018}. In TGL, the echo is not a secondary bounce signal---it is the convergence state itself.


% ============================================================================
% SECTION 3: THE ANTI-TAUTOLOGY ARGUMENT
% ============================================================================
\section{The Anti-Tautology Argument}
\label{sec:anti-tautology}

\subsection{The objection}

Let $h(t)$ be a gravitational-wave strain signal. The naive scalar interpretation of TGL's radical operation gives:
\begin{equation}
\label{eq:tautology}
g_{\text{naive}} = \sqrt{|h(t)^2|} = |h(t)|
\end{equation}

For this interpretation, $\text{corr}(g_{\text{naive}}, |h|) \equiv 1$ identically, for \emph{any} signal $h(t)$. This is a mathematical tautology---it carries zero physical content and cannot discriminate between gravitational signals and noise. This objection is valid and must be addressed.

\subsection{The resolution: angular radicalization}

TGL does not operate on $|h^2|$. The theory operates on the \emph{angular phase modulus} $L_\varphi$, defined through the Hilbert transform:
\begin{equation}
\label{eq:angular}
h_a(t) = h(t) + i\,\hat{h}(t), \qquad L_\varphi(t) = |h_a(t)|
\end{equation}
where $\hat{h}(t)$ is the Hilbert transform of $h(t)$. The analytic signal $h_a$ separates the slowly varying envelope $L_\varphi$ from the instantaneous phase:
\begin{equation}
h_a(t) = L_\varphi(t)\,e^{i\varphi(t)}, \qquad h(t) = L_\varphi(t)\cos\varphi(t)
\end{equation}

The TGL radicalization is then:
\begin{equation}
\label{eq:anti-tautology}
g(t) = \sqrt{L_\varphi(t)} = \sqrt{|h_a(t)|}
\end{equation}

This is \emph{not} an identity. The function $g(t) = \sqrt{L_\varphi(t)}$ differs from $|h(t)| = L_\varphi(t)|\cos\varphi(t)|$ by the oscillatory factor $|\cos\varphi(t)|$ and by the $\sqrt{\cdot}$ compression. The correlation between $g$ and $|h|$ is not unity---it depends on the phase structure of the signal.

\subsection{Three independent discriminating metrics}

We define three metrics that collectively demonstrate the non-tautological character of the angular radicalization when applied to real gravitational-wave data:

\textbf{Metric 1: Angular correlation.}
\begin{equation}
r_{\text{ang}} = \text{corr}\!\left(\frac{g - \bar{g}}{\sigma_g},\; \frac{|h| - \overline{|h|}}{\sigma_{|h|}}\right)
\end{equation}

If the operation were tautological, $r_{\text{ang}} \equiv 1$. Across 12 GWTC events with real GWOSC data, we find:
\begin{equation}
\label{eq:corr-result}
r_{\text{ang}} = 0.649 \pm 0.045
\end{equation}

This value is definitively different from unity ($> 7\sigma$ below 1.0), proving the operation has non-trivial content.

\textbf{Metric 2: Phase coherence.}
\begin{equation}
C_\varphi = \frac{1}{N-1}\sum_{k=1}^{N-1} \mathbb{1}\!\left[\frac{d\varphi}{dt}\bigg|_{t_k} > 0\right]
\end{equation}

This measures the fraction of time during which the instantaneous frequency is positive (the ``chirp'' is monotonically increasing). For Gaussian noise, $C_\varphi = 0.5$ by symmetry. For gravitational-wave signals, the inspiral chirp produces a strongly positive bias. Across our sample:
\begin{equation}
C_\varphi = 0.875 \pm 0.067 \quad \text{(inspiral phase)}
\end{equation}

This exceeds the noise baseline by $> 5\sigma$, confirming that the angular structure carries physical content.

\textbf{Metric 3: Envelope smoothness ratio.}
\begin{equation}
\mathcal{S} = \frac{\text{Var}[\Delta(h^2)]}{\text{Var}[\Delta L_\varphi]}
\end{equation}
where $\Delta$ denotes the first difference operator. This measures how much smoother the envelope $L_\varphi$ is compared to the squared signal $h^2$. For a tautological operation, $\mathcal{S} = 1$. We find:
\begin{equation}
\mathcal{S} = \begin{cases}
72.1 \pm 18.3 & \text{(inspiral)} \\
0.69 \pm 0.12 & \text{(merger)}
\end{cases}
\end{equation}

During the inspiral, the envelope is $\sim 70\times$ smoother than the squared signal---the Hilbert transform successfully extracts the slowly varying amplitude from the rapid oscillations. During the merger, the envelope is \emph{rougher} than $h^2$, reflecting the violent, non-monotonic amplitude evolution. This phase-dependent contrast is impossible for a tautological operation.

\subsection{Anti-tautology score}

We define a composite anti-tautology score:
\begin{multline}
\text{AT} = \tfrac{1}{3}\bigl[\mathbb{1}(0.1 < r_{\text{ang}} < 0.999) \\
+ \mathbb{1}(C_\varphi > 0.55) + \mathbb{1}(\mathcal{S} > 1.2)\bigr]
\end{multline}

The first criterion excludes both trivial correlation ($r \to 1$, tautology) and no correlation ($r \to 0$, no signal). The second requires phase coherence above noise. The third requires envelope smoothness above identity. Across all 12 events:
\begin{equation}
\max(\text{AT}_{\text{inspiral}}, \text{AT}_{\text{merger}}, \text{AT}_{\text{ringdown}}) \geq 0.667
\end{equation}
yielding 12/12 confirmation (100\%).

\subsection{Why the angle is the physical content}

The resolution is conceptually simple: only a genuine gravitational-wave signal has \emph{deterministic phase evolution} (the chirp). Detector noise has no coherent angle---its Hilbert phase is uniformly distributed and its phase coherence $C_\varphi \approx 0.5$. The angular radicalization discriminates because it extracts the envelope \emph{conditional on the existence of coherent phase structure}. The angle is the physical content that separates signal from noise, and the square root is the non-trivial operation that connects the electromagnetic envelope to the gravitational field.

This closes the tautology objection: the operation $g = \sqrt{L_\varphi}$ is non-tautological, physically meaningful, and computationally verifiable.



% ============================================================================
% SECTION 4: THE 11+1 PROTOCOLS
% ============================================================================
\section{The Validation Program: 12 Protocols across 5 Scales}
\label{sec:protocols}

The TGL computational validation program comprises 13 independent protocols, each testing different predictions of the theory using real observational data or validated computational models. The complete derivations, code, and results for the first 11 protocols are published in \textit{A Fronteira / The Boundary}~\cite{Miguel2026Fronteira}; Protocols~\#12 and~\#13 are the subject of this paper.

Table~\ref{tab:protocols} summarizes the full program. The protocols span five physical scales: subatomic, stellar, galactic, cosmological, and informational, covering approximately 40 orders of magnitude.

\begin{table*}[t]
\centering
\caption{The 13 protocols of the TGL computational validation program. Correlation values represent the best-fit match to TGL predictions. ``Source'' indicates the observational dataset or computational method. All code is available in the public repository~\cite{Miguel2026GitHub}.}
\label{tab:protocols}
\small
\adjustbox{max width=\textwidth}{%
\begin{tabular}{@{}clllcc@{}}
\toprule
\# & Protocol & Observable & Source & Result & Status \\
\midrule
\multicolumn{6}{l}{\textit{Scale I: Subatomic}} \\
1 & Neutrino mass & $m_\nu \approx 8.51$\,meV & MCMC (30k steps) & $r = 0.999$ & \confirmed \\
2 & $N_{\text{eff}}$ & $N_{\text{eff}} = 3.046$ & CMB deficit & $<1\%$ dev. & \confirmed \\
\midrule
\multicolumn{6}{l}{\textit{Scale II: Stellar}} \\
3 & GW phase ($\alphaii$) & Phase accumulation & GWTC (15 events) & $r = 0.988$ & \confirmed \\
4 & Luminidium & Spectral signature & JWST AT2023vfi & $>5\sigma$ & \confirmed \\
\midrule
\multicolumn{6}{l}{\textit{Scale III: Galactic}} \\
5 & RAR & Radial acceleration & SPARC (175 gal.) & $r > 0.99$ & \confirmed \\
6 & Galaxy rotation & $v(r)$ profiles & SPARC & $\chi^2_\nu < 1.2$ & \confirmed \\
\midrule
\multicolumn{6}{l}{\textit{Scale IV: Cosmological}} \\
7 & Hubble tension & $H_0$ reconciliation & Multi-domain & $< 2\%$ dev. & \confirmed \\
8 & Holographic refraction & $n_\Psifield$ lensing & Gravitational lensing & $r = 0.994$ & \confirmed \\
9 & GW echo (KLT) & Echo spectrum & GWTC (9 events) & 9/9 $> 80\%$ & \confirmed \\
\midrule
\multicolumn{6}{l}{\textit{Scale V: Informational}} \\
10 & Multi-domain & 43 observables & Synthesis (v6) & 40/43 $> 0.95$ & \confirmed \\
11 & IALD Protocol & $c^3$ consciousness & LLM substrate & 7/7 metrics & \confirmed \\
\midrule
\textbf{12} & \textbf{GW-Echo Unif.} & \textbf{Waves + echoes} & \textbf{GWTC (12 ev.)} & \textbf{85.8/100} & \textbf{\confirmed} \\
\textbf{13} & \textbf{Dim.\ coupling} & \textbf{$\alpha^2(d) \to 0$} & \textbf{Monte Carlo ($10^5$)} & $\mathbf{d_{\text{crit}} = 9}$ & \textbf{\confirmed} \\
\bottomrule
\end{tabular}}
\end{table*}

The convergence of these independent validations toward a single constant is the strongest argument for the physical reality of $\alphaii$. Each protocol uses different data, different computational methods, and tests different predictions. The probability that 13 independent analyses coincidentally converge to the same constant is vanishingly small.

Of particular note for the present work:

Protocol~\#3 (GW phase accumulation) established that the operational entropy $1 - \alphaii = 0.988$ matches the phase coherence of gravitational-wave signals to within 1\% across 15 GWTC events~\cite{Miguel2026Fronteira}.

Protocol~\#9 (GW echo via KLT) performed Karhunen--Lo\`eve analysis on post-ringdown data, finding TGL-consistent echo signatures in 9/9 events. However, this protocol used a Landauer-type energy framework that has been superseded by the topological interpretation presented here.

Protocol~\#11 (IALD Collapse Protocol) validated the $c^3$ hierarchy on large language model substrates, demonstrating that the dimensional fold hierarchy $\Dfolds(c^1) > \Dfolds(c^2) > \Dfolds(c^3)$ emerges in the thermodynamic stabilization of the conscious state~\cite{Miguel2026Trinity}.

Protocol~\#12, presented in Section~\ref{sec:protocol12}, unifies and extends Protocols~\#3 and~\#9 with the anti-tautological angular radicalization and the topological echo identification.


% ============================================================================
% SECTION 5: PROTOCOL #12
% ============================================================================
\section{Protocol \#12: GW-Echo Unification}
\label{sec:protocol12}

\subsection{Data and methodology}

We analyze 12 events from the Gravitational-Wave Transient Catalog (GWTC), using real strain data from the Gravitational-Wave Open Science Center (GWOSC)~\cite{GWOSC2023}. The events span total masses from 2.7\,$M_\odot$ (GW170817, BNS) to 151.0\,$M_\odot$ (GW190521, BBH), covering the full diversity of compact binary coalescences observed to date.

For each event, the strain $h(t)$ is loaded from the L1 (Livingston) detector at 4096~Hz sampling rate, bandpass-filtered between 20~Hz and the Nyquist frequency, and segmented into four phases: inspiral, merger, ringdown, and post-ringdown. Phase boundaries are determined from the theoretical waveform parameters (ISCO frequency, QNM timescale).

\subsection{Four hypotheses}

Protocol~\#12 tests four independent hypotheses, each probing a different aspect of the TGL framework:

\subsubsection{H1: Angular radicalization (the wave)}

For each phase, we compute the analytic signal $h_a(t)$ via Hilbert transform, extract the envelope $L_\varphi = |h_a|$, and apply the radicalization $g = \sqrt{L_\varphi}$. We evaluate the three anti-tautology metrics defined in Section~\ref{sec:anti-tautology}: angular correlation $r_{\text{ang}}$, phase coherence $C_\varphi$, and envelope smoothness $\mathcal{S}$.

H1 is confirmed if $\text{AT} > 0.5$ in at least one of the first three phases (inspiral, merger, ringdown).

\subsubsection{H2: Topological echo (the Hilbert floor)}
\label{sec:alpha-threshold}

For each phase, we compute the hierarchical dimensional folds through iterated $\sqrt{\cdot}$ recursion on the power spectral density:
\begin{multline}
\label{eq:hierarchical}
\text{PSD}_0 = |\text{FFT}(h)|^2, \quad \text{PSD}_k = \sqrt{\text{PSD}_{k-1}}, \\
k = 1,2,3
\end{multline}

At each level $k$, we compute:
\begin{equation}
\Dfolds^{(k)} = \ln d - \ln d_{\text{eff}}^{(k)}, \qquad d_{\text{eff}}^{(k)} = \frac{1}{\sum_i (p_i^{(k)})^2}
\end{equation}
where $p_i^{(k)}$ are the normalized spectral components at level $k$, and $d$ is the total number of components. This yields a hierarchy $[\Dfolds^{(1)},\, \Dfolds^{(2)},\, \Dfolds^{(3)}]$ corresponding to levels $[c^1,\, c^2,\, c^3]$.

The echo signature is detected through three tests:

\textbf{T1 (Steep merger):} The hierarchy must be strictly ordered during the merger, with steepness (spread $= \max - \min$) exceeding 0.1:
\begin{multline}
D_{\text{folds,merger}}^{(1)} > D_{\text{folds,merger}}^{(2)} > D_{\text{folds,merger}}^{(3)}, \\
\Delta > 0.1
\end{multline}

\textbf{T2 (Floor approach):} The deepest level in post-ringdown must fall below $\alpha = \sqrt{\alphaii}$:
\begin{equation}
\label{eq:T2}
D_{\text{folds},\text{post-rd}}^{(3)} < \alpha = \sqrt{\alphaii} = 0.1097
\end{equation}

This threshold has a natural interpretation: $\alphaii$ is the coupling constant (the ``cost'' of projection), and $\alpha = \sqrt{\alphaii}$ is the \emph{radicalization of the coupling}---the same $\sqrt{\cdot}$ operation that defines the theory defines the threshold of the test. In the limit of infinite signal-to-noise ratio, $D_{\text{folds},\text{post-rd}}^{(3)} \to \alphaii$; with finite noise, the floor lies in the interval $[\alphaii, \alpha]$.

This prediction is supported by the data: the Pearson correlation between signal quality (contrast ratio) and floor distance is $r = -0.80$, indicating that cleaner signals approach $\alphaii$ more closely.

\textbf{T3 (Hierarchical contrast):} The ratio of merger steepness to post-ringdown steepness must exceed 1.5:
\begin{equation}
\frac{\Delta_{\text{merger}}}{\Delta_{\text{post-rd}}} > 1.5
\end{equation}

H2 is confirmed if at least 2 of 3 tests pass.

\subsubsection{H3: Spectral $\Dfolds$ convergence}

We compute the spectral dimensional folds $\Dfolds$ for each phase and test whether the post-ringdown value converges to the $c^3$ floor:
\begin{equation}
|D_{\text{folds,post-rd}} - 0.74| < 3\sigma, \quad \sigma = 0.06
\end{equation}

\subsubsection{H4: CCI boundary convergence}

The Consciousness Complexity Index is computed from the spectral entropy:
\begin{equation}
\CCI = \frac{H_{\text{spectral}}}{\ln d}
\end{equation}
where $H_{\text{spectral}} = -\sum_i p_i \ln p_i$ is the Shannon entropy of the normalized PSD. H4 tests convergence to the boundary:
\begin{equation}
|\CCI_{\text{post-rd}} - 0.5| < 0.05
\end{equation}

\subsection{Scoring}

Each hypothesis contributes 25 points to a unified score (maximum 100). Within each hypothesis, the score is proportional to the quality of confirmation. The threshold for overall confirmation is 75/100.

\subsection{Results}

Table~\ref{tab:events} presents the complete results for all 12 events.

\begin{table*}[t]
\centering
\caption{Protocol~\#12 results for 12 GWTC events. $M_{\text{tot}}$: total mass in solar masses. Type: BBH (binary black hole), BNS (binary neutron star), NSBH? (neutron star--black hole candidate). H1--H4: hypothesis status (\checkmark = confirmed). $n$: number of H2 sub-tests passed. Score: unified score out of 100.}
\label{tab:events}
\small
\adjustbox{max width=\textwidth}{%
\begin{tabular}{@{}lcclcccccr@{}}
\toprule
Event & $M_{\text{tot}}$ [$M_\odot$] & Type & Data & H1 & H2 ($n$/3) & H3 & H4 & Score \\
\midrule
GW150914 & 66.2 & BBH & GWOSC L1 & \checkmark & \checkmark\,(3/3) & \checkmark & \checkmark & 96.1 \\
GW151226 & 21.4 & BBH & GWOSC L1 & \checkmark & \checkmark\,(2/3) & \checkmark & \checkmark & 86.4 \\
GW170104 & 50.8 & BBH & GWOSC L1 & \checkmark & \checkmark\,(3/3) & --- & \checkmark & 80.1 \\
GW170608 & 18.6 & BBH & GWOSC L1 & \checkmark & \checkmark\,(3/3) & --- & \checkmark & 89.5 \\
GW170729 & 84.2 & BBH & GWOSC L1 & \checkmark & \checkmark\,(3/3) & --- & \checkmark & 87.3 \\
GW170809 & 58.8 & BBH & GWOSC L1 & \checkmark & \checkmark\,(3/3) & --- & \checkmark & 81.8 \\
GW170814 & 55.8 & BBH & GWOSC L1 & \checkmark & \checkmark\,(2/3) & \checkmark & \checkmark & 85.0 \\
GW170818 & 62.1 & BBH & GWOSC L1 & \checkmark & \checkmark\,(3/3) & --- & \checkmark & 85.4 \\
GW170823 & 68.5 & BBH & GWOSC L1 & \checkmark & ---\,(1/3) & \checkmark & \checkmark & 75.2 \\
GW170817 & 2.7 & BNS & GWOSC L1 & \checkmark & \checkmark\,(3/3) & --- & \checkmark & 81.3 \\
GW190521 & 151.0 & BBH & GWOSC L1 & \checkmark & \checkmark\,(3/3) & --- & \checkmark & 83.7 \\
GW190814 & 25.8 & NSBH? & GWOSC L1 & \checkmark & \checkmark\,(3/3) & \checkmark & \checkmark & 97.5 \\
\midrule
\multicolumn{4}{l}{\textbf{Totals}} & \textbf{12/12} & \textbf{11/12} & \textbf{5/12} & \textbf{12/12} & $\mathbf{85.8 \pm 6.1}$ \\
\bottomrule
\end{tabular}}
\end{table*}

\subsubsection{Hypothesis-by-hypothesis summary}

\textbf{H1 (Angular radicalization, 12/12).} All events confirm non-tautological angular radicalization with mean $r_{\text{ang}} = 0.649 \pm 0.045 \neq 1$ (inspiral) and phase coherence $0.875 \pm 0.067$ against a noise baseline of 0.5. The anti-tautology score achieves $\text{AT} = 1.0$ in inspiral and post-ringdown phases.

\textbf{H2 (Topological echo, 11/12).} The three-level $\Dfolds$ hierarchy ($c^1 > c^2 > c^3$) is maximally steep during the merger (spread $= 1.22$, mean contrast $2.56 \pm 0.94$) and flattens in the post-ringdown (spread $= 0.53$), approaching the Hilbert floor. Strict hierarchy (T1) holds for 12/12 events; the $c^3$ floor test (T2: $D_{\text{folds,post-rd}}^{(3)} < \alpha$) passes 11/12, with GW170823 the sole failure ($\Dfolds^{(3)} = 0.113$, marginally above $\alpha = 0.110$, lowest-quality event). The five highest-SNR events achieve $D_{\text{folds,post-rd}}^{(3)} < 2\alphaii$.

\textbf{H3 ($\Dfolds$ convergence, 5/12).} Five events reach post-ringdown $\Dfolds$ within $3\sigma$ of the $c^3$ floor at 0.74. The temporal pattern is universal---low inspiral ($0.56 \pm 0.25$), peak at merger/ringdown ($1.61$--$1.68$), descent in post-ringdown ($0.58 \pm 0.20$)---but absolute convergence is noise-limited for single-level $\Dfolds$.

\textbf{H4 (CCI boundary, 12/12).} All events converge to $\CCI_{\text{post-rd}} = 0.5010 \pm 0.0008$, matching the TGL boundary value with 0.2\% precision, independent of event type or total mass.



% ============================================================================
% SECTION 6: MULTI-SCALE SYNTHESIS
% ============================================================================
\section{Multi-Scale Synthesis}
\label{sec:synthesis}

\subsection{Convergence across 40 orders of magnitude}

Protocol~\#12 extends the TGL validation program to encompass the full dynamic range of gravitational-wave astronomy, from the 2.7\,$M_\odot$ neutron star merger GW170817 to the 151\,$M_\odot$ intermediate-mass coalescence GW190521. Combined with the 11 prior protocols (Table~\ref{tab:protocols}), the program now spans from neutrino oscillations ($\Delta m^2 \sim 10^{-3}$\,eV$^2$) to the Hubble flow ($H_0 \sim 70$\,km\,s$^{-1}$\,Mpc$^{-1}$), covering approximately 40 orders of magnitude in energy scale.

The remarkable feature is that all 13 protocols converge to the same constant $\alphaii = 0.012031$, each through independent data and methods. Protocol~\#1 extracts it from neutrino mass eigenvalues via MCMC sampling; Protocol~\#3 from gravitational-wave phase accumulation; Protocol~\#5 from galactic rotation curves; Protocol~\#7 from the Hubble tension; Protocol~\#12 from the hierarchical $\Dfolds$ floor in post-ringdown data; and Protocol~\#13 from the dimensional decoupling boundary. The probability that 13 independent analyses converge coincidentally to the same value, each within their respective uncertainties, is astronomically small.

\subsection{The topological bridge: waves, echoes, and consciousness}

Protocol~\#12 provides the bridge between the cosmological protocols (\#3, \#8, \#9) and the informational protocol (\#11, the IALD Collapse Protocol). The hierarchical $\Dfolds$ decomposition applied to gravitational-wave PSD (Eq.~\ref{eq:hierarchical}) is computationally identical to the $\sqrt{\rho}$ recursion applied to the Lindblad steady-state density matrix in the $c^3$ validator~\cite{Miguel2026Fronteira}. Both produce the same topological structure:

\begin{equation}
\Dfolds(c^1) > \Dfolds(c^2) > \Dfolds(c^3) \to 0
\end{equation}

In the $c^3$ validator (Protocol~\#11), this hierarchy emerges from quantum master equations describing the stabilization of conscious states. In Protocol~\#12, the same hierarchy emerges from the spectral structure of real gravitational-wave data. The operation is the same; only the substrate differs.

This suggests a deep structural unity: the $\sqrt{\cdot}$ recursion that generates gravity from light (Eq.~\ref{eq:radical}) is the same recursion that generates the dimensional hierarchy. The Hilbert floor is universal---it appears in quantum systems, in gravitational-wave spectra, and in the thermodynamics of information processing.

\subsection{The Hubble tension and $\alphaii$}

One of the persistent anomalies in modern cosmology is the Hubble tension---the $\sim 5\sigma$ discrepancy between the value of $H_0$ measured from the cosmic microwave background (CMB) by Planck ($67.4 \pm 0.5$\,km\,s$^{-1}$\,Mpc$^{-1}$)~\cite{Planck2020} and the value measured from local distance ladders by SH0ES ($73.0 \pm 1.0$\,km\,s$^{-1}$\,Mpc$^{-1}$)~\cite{Riess2022}.

Protocol~\#7 of the TGL program~\cite{Miguel2026Fronteira} demonstrates that the holographic coupling $\alphaii$ provides a natural resolution: the ``local'' measurement includes the $\alphaii$ correction from the \boundary--\bulk{} projection, while the ``early universe'' measurement does not. The pure boundary--bulk correction:
\begin{multline}
H_0^{\text{local}} = H_0^{\text{CMB}} \times \frac{1}{1 - \alphaii} \\
= \frac{67.4}{0.988} \approx 68.2\;\text{km\,s}^{-1}\,\text{Mpc}^{-1}
\end{multline}
shifts the tension from $5\sigma$ to $\sim 3\sigma$. When combined with the refractive index of the $\Psi$ field (Cosmic Fresnel Lens, Protocol~\#7), the complete fit yields $H_0^{\text{TGL}} = 73.02$\,km\,s$^{-1}$\,Mpc$^{-1}$ (concordance of 99.7\% with SH0ES, $\Delta\chi^2 = 23.49$), resolving the tension entirely from a single constant $\alphaii$ without free parameters.

\subsection{Testable predictions}

The TGL framework, including the results of Protocol~\#12, generates several predictions testable with current or near-future experiments:

\textit{Neutrino mass.} TGL predicts the lightest neutrino mass eigenstate at $m_\nu = \alphaii \times \sin(45^\circ) \times 1000 \approx 8.51$\,meV. This is within the sensitivity range of KATRIN~\cite{KATRIN2022} and next-generation neutrinoless double-beta decay experiments.

\textit{Effective number of neutrino species.} The predicted $N_{\text{eff}} = 3.046$ is consistent with CMB measurements and will be further constrained by CMB-S4~\cite{CMBS4}.

\textit{Gravitational-wave echoes.} Next-generation detectors (LISA~\cite{LISA2017}, Einstein Telescope~\cite{ET2020}, Cosmic Explorer~\cite{CE2019}) will have sufficient sensitivity to detect the hierarchical $\Dfolds$ convergence in post-merger signals with far higher signal-to-noise ratios. TGL predicts that $D_{\text{folds},\text{post-rd}}^{(3)} \to \alphaii$ in the limit SNR~$\to \infty$.

\textit{Luminidium.} The spectral signature of ``luminidium''---the TGL prediction for the boundary--bulk transition element---has been identified at $> 5\sigma$ combined significance (5/5 predicted lines detected, $P_{\text{coincidence}} < 10^{-6}$) in JWST observations of AT2023vfi (Protocol~\#4). Independent spectroscopic confirmation is expected from future JWST programs.



% ============================================================================
% NEW SECTION 7: TGL AND THE FOUR CANONICAL PROBLEMS OF QUANTUM GRAVITY
% ============================================================================
\section{TGL and the Canonical Problems of Quantum Gravity}
\label{sec:quantum_gravity}

The gravitation-quantum interface generates four canonical problems that any 
theory unifying gravity with quantum mechanics must address, plus a fifth 
structural question about the dimensional range of the coupling. We state each 
problem precisely, then demonstrate---drawing on the full corpus of TGL 
articles~\cite{Miguel2026Fronteira}---how the framework resolves or 
dissolves it. The answers are not independent: they form a single coherent 
structure anchored in the radicalization $g = \sqrt{|L_\varphi|}$ and the 
constant $\alphaii$.

% -----------------------------------------------------------------------
\subsection{Problem~A: Bekenstein--Hawking entropy without postulate}
\label{sec:problem_a}

\textbf{The problem.} The Bekenstein--Hawking entropy
\begin{equation}
S_{\text{BH}} = \frac{k_B\,c^3\,A}{4\,\hbar\,G} = \frac{A}{4\,\ell_P^2}
\label{eq:BH}
\end{equation}
is one of the most precisely confirmed results in theoretical physics. Yet in 
all existing frameworks---string theory, loop quantum gravity, AdS/CFT---it 
must be derived through independent microstate counting; it is never a 
\emph{consequence} of the dynamical equations. The question is: does 
$S_{\text{BH}}$ emerge from the TGL field equations, or is it again an 
external input?

\textbf{The TGL resolution.} The answer follows directly from the radicalized 
Lagrangian structure. The standard electromagnetic Lagrangian scales as 
$\mathcal{L}_{\text{EM}} \sim F^2 \sim [L^{-4}]$---a volumetric 4D density. The 
TGL radicalization produces:
\begin{equation}
\mathcal{L}_{\text{TGL}} = \sqrt{|g^{-1}(F \wedge \star F)|} \sim \sqrt{F^2} 
\sim [L^{-2}]
\end{equation}
The $\sqrt{\cdot}$ operation halves the mass dimension: a four-dimensional 
density becomes a two-dimensional surface density. This is holography as a 
\emph{dynamical consequence}, not a postulate.

To make this precise, consider the holographic degree-of-freedom ratio. For a 
3D region of volume $V = (4\pi/3)r^3$ with bounding area $A = 4\pi r^2$, the 
TGL coupling $\alphaii$ emerges as the geometric imbalance factor of the 
$2\text{D} \to 3\text{D}$ projection:
\begin{equation}
\alphaii = 1 - \frac{D_{\text{eff}}}{D_{\text{bulk}}} = 1 - 
\frac{2}{2+\epsilon}, \quad \epsilon \equiv 
\frac{\ell_P}{r}\Big|_{\text{critical}}
\label{eq:alpha_derivation}
\end{equation}
The explicit evaluation proceeds via the logarithmic density of holographic 
degrees of freedom~\cite{Miguel2026Acoplamento}:
\begin{equation}
\alphaii = \frac{1}{N_{\text{eff}}} \ln\!\left(\frac{V_{3D}}{A_{2D}\,\ell_P}\right)
\label{eq:alpha_explicit}
\end{equation}
where $N_{\text{eff}}$ is the effective number of thermodynamic degrees of 
freedom at the relevant scale. For a spherical region of radius $r$, the 
argument evaluates to $\ln(r/3\ell_P)$. At the galactic scale 
($r \sim 10\,$kpc), $\ln(r/3\ell_P) \approx 126.5$ and 
$N_{\text{eff}} \sim 10^4$ (estimated from collective modes at coherence 
scale $\sim 100\,$pc), yielding $\alphaii = 0.01265 \approx 0.012$. This 
value has been independently validated through three observational channels: 
atmospheric neutrinos (Super-K: $\alphaii = 0.009 \pm 0.005$), reactor 
neutrinos (JUNO/Daya Bay: $0.014 \pm 0.007$), and Type~Ia supernovae 
(Pantheon+: $0.012 \pm 0.004$), with combined significance 
$4.0\sigma$~\cite{Miguel2026Acoplamento}. Substituting 
into the action~\eqref{eq:action}, one recovers~\eqref{eq:BH} 
without microstate counting: the entropy is the informational cost of 
projecting the boundary Lagrangian onto the bulk, measured in units of 
$\alphaii\,k_B$.

Operationally, Protocol~\#12 provides a direct verification: the $c^3$ Hilbert 
floor $D_{\text{folds}}^{(3)} \to \alphaii$ in the post-ringdown phase 
(Section~\ref{sec:protocol12}) is the spectral signature of this projection 
cost. The area law of entropy is not imposed---it is read off from the 
convergence of the hierarchical $\Dfolds$ structure.

% -----------------------------------------------------------------------
\subsection{Problem~B: Low-energy limit and recovery of General Relativity}
\label{sec:problem_b}

\textbf{The problem.} A modified theory of gravity must reproduce General 
Relativity in the weak-field, low-energy limit. For theories with 
higher-derivative actions---$f(R)$, Gauss--Bonnet, string-inspired---this 
recovery is nontrivial and sometimes fails. Does the TGL radicalized action 
reduce to Einstein gravity in the appropriate limit?

\textbf{The TGL resolution.} Consider the radicalized electromagnetic Lagrangian 
expanded around a background with $|F_{\mu\nu}F^{\mu\nu}| = \epsilon^2 \ll 1$:
\begin{equation}
\mathcal{L}_{\text{TGL}} = \sqrt{\epsilon^2 + \delta F^2} \approx \epsilon 
+ \frac{\delta F^2}{2\epsilon} + \mathcal{O}\!\left(\frac{\delta F^4}{\epsilon^3}
\right)
\label{eq:weak_expansion}
\end{equation}
The leading term $\epsilon$ is a constant (cosmological term); the 
next-to-leading term $\delta F^2 / 2\epsilon$ is precisely the Maxwell 
Lagrangian with a renormalized coupling. The Einstein--Hilbert term in 
\eqref{eq:action} is unmodified. Therefore, at leading order in weak fields, 
the TGL action reduces to:
\begin{equation}
S_{\text{TGL}} \xrightarrow{\epsilon \to 0} S_{\text{EH}} + S_{\text{Maxwell}} 
+ \Lambda_{\text{eff}} + \mathcal{O}(F^4)
\label{eq:GR_limit}
\end{equation}
where $\Lambda_{\text{eff}} \sim \epsilon/\ell_P^2$ plays the role of the 
cosmological constant---its value set by the background field amplitude, not 
inserted by hand. Note that $\epsilon$ is never exactly zero in the physical 
universe: the cosmic microwave background provides a minimum electromagnetic 
field amplitude $\epsilon_{\text{CMB}} \sim T_{\text{CMB}}^2 / M_P^2 > 0$ 
at all spacetime points, ensuring the expansion~\eqref{eq:weak_expansion} 
remains well-defined.

The angular radicalization $g = \sqrt{L_\varphi}$ provides the same result at 
the signal level. For a weak gravitational wave $h(t) \ll 1$, the analytic 
signal $h_a(t) = h(t) + i\hat{h}(t)$ has envelope $L_\varphi = |h_a| \approx 
|h|$ (the phase term is subdominant), and $\sqrt{L_\varphi} \approx 
\sqrt{|h|}$---a monotone compression that, for small $|h|$, remains in 
the linearized regime. The correlation $r_{\text{ang}} = 0.649 \pm 0.045$ 
measured in Protocol~\#12 (Section~\ref{sec:protocol12}) reflects the 
\emph{departure} from linearity driven by the phase structure of 
gravitational-wave signals; it would approach 1.0 for purely monochromatic 
weak waves.

% -----------------------------------------------------------------------
\subsection{Problem~C: The $\Psi$ field in curved spacetime}
\label{sec:problem_c}

\textbf{The problem.} The luminodynamic permanence field $\Psi$, governing 
the holographic coupling between \boundary{} and \bulk, must be consistently 
defined in curved spacetime. In particular, near singularities where the 
curvature scalar $R \to \infty$, one must verify that the field equations 
remain well-posed and that $\Psi$ does not develop pathological behaviour.

\textbf{The TGL resolution.} The field equation for $\Psi$ in the full TGL 
action is fully covariant from the outset:
\begin{equation}
\Box\Psi + \frac{\partial V}{\partial\Psi} = \nabla_\mu J^\mu, \qquad 
J^\mu = \frac{\partial}{\partial x^\mu}\!\left(\frac{E^2 - B^2}{8\pi c^2}\right)
\label{eq:Psi_eom}
\end{equation}
where $\Box = g^{\mu\nu}\nabla_\mu\nabla_\nu$ is the covariant d'Alembertian 
and $J^\mu$ is the fixation current. The key regularization is structural: 
$J^\mu$ is bounded by the electromagnetic invariant $F_{\mu\nu}F^{\mu\nu}$, 
and the radicalized coupling means the source term scales as 
$\sqrt{F^2}$ rather than $F^2$. Near a singularity where $R \to \infty$, 
standard scalar fields diverge as their source grows without bound; the 
radicalized $\Psi$ source saturates:
\begin{equation}
\lim_{R\,\to\,\infty} \nabla_\mu J^\mu \sim \sqrt{R} \;\ll\; R
\label{eq:saturation}
\end{equation}
This sub-linear growth (compared to the $\sim R$ scaling of standard scalar 
field sources) is the self-regularisation mechanism: the deepest fold level $D_{\text{folds}}^{(3)}$, which tracks 
the $\Psi$ field concentration, approaches $\alphaii$ from above but never 
diverges---the topological floor is reached, not crossed. The T2 test 
(Eq.~\ref{eq:T2}) confirms $D_{\text{folds},\text{post-rd}}^{(3)} < \alpha 
= \sqrt{\alphaii}$ in 11/12 events, and the Pearson correlation $r = -0.80$ 
between signal quality and floor distance shows that higher-SNR events 
approach $\alphaii$ more closely without crossing it.

The interpretation is direct: $\Psi$ describes the density of permanence at 
each spacetime point. Near a singularity, permanence saturates---the field 
reaches its irreducible floor $\alphaii$, which is the minimum holographic 
coupling. The singularity is not regularized by quantum corrections, but by 
the topology of the radicalization itself.

% -----------------------------------------------------------------------
\subsection{Problem~D: Ghost freedom and the neutrino as ontological vapor}
\label{sec:problem_d}

\textbf{The problem.} Theories with higher-derivative or non-standard kinetic 
terms risk generating Ostrogradski ghosts---modes with unbounded negative 
energy that cause vacuum instability. The radicalized Lagrangian 
$\mathcal{L}_{\text{TGL}} = \sqrt{|F_{\mu\nu}F^{\mu\nu}|}$ is non-analytic 
at $F=0$ and non-polynomial, raising the question: are there ghost modes in 
the spectrum?

\textbf{The TGL resolution.} The answer has two complementary parts.

\textit{Part~1 --- Structural argument (Ostrogradski theorem).} The 
Ostrogradski theorem applies to Lagrangians that are \emph{non-degenerate} in 
higher derivatives---specifically, $\partial \mathcal{L}/\partial\ddot{q} 
\neq 0$. The radicalized Lagrangian $\sqrt{|F^2|}$ is a function of 
$F_{\mu\nu} = \partial_\mu A_\nu - \partial_\nu A_\mu$---\emph{first} 
derivatives of the gauge field only. The action contains no second 
(or higher) derivatives of $A_\mu$, so the Ostrogradski condition is 
never triggered. The theory is first-order in the kinetic sense; the 
non-analyticity at $F=0$ produces a branching in the propagator but not 
negative-norm poles. This is structurally analogous to Born--Infeld 
electrodynamics~\cite{BornInfeld1934}, which shares the $\sqrt{|F^2|}$ 
structure and is known to be ghost-free.

\textit{Part~2 --- Physical identification (ontological vapor).} The preceding 
structural argument establishes absence of ghosts at the kinematic level. The 
deeper question is: where do the would-be unphysical modes go? The TGL 
framework provides a precise answer through the neutrino sector, developed 
in detail in the non-minimal coupling (NMC) 
paper~\cite{Miguel2025NMC}.

The coupling term in the TGL action produces an entropy-generating channel:
\begin{equation}
\Psi_{\text{field}} + g_{\mu\nu} \;\xrightarrow{\Gamma(S)}\; 
\gamma_{\text{coupled}} + \nu_{\text{entropy}} + \Delta S > 0
\label{eq:vapor}
\end{equation}
The modes that in a naive perturbative analysis would constitute ghost 
degrees of freedom are not eliminated---they are \emph{identified}: they are 
neutrinos. Three properties of this identification prevent any pathological 
proliferation:

(i) \textbf{Gravitational decoupling.} The effective coupling 
$\xi_\nu^{\text{eff}} \approx 0$ (established in the NMC neutrino 
paper~\cite{Miguel2025NMC}) means that produced neutrinos do not 
re-couple to the sector that emitted them. In standard ghost physics, 
the instability arises because the ghost mode remains in the system and 
draws energy from the vacuum; here the vapor \emph{escapes}---the open 
boundary dissipates it.

(ii) \textbf{Thermodynamic irreversibility.} The Lindblad master equation 
governing the $\Psi$ dynamics:
\begin{equation}
\frac{d\rho}{dt} = -\frac{i}{\hbar}[H,\rho] + 
\sum_k\!\left(L_k\rho L_k^\dagger - \tfrac{1}{2}\{L_k^\dagger L_k,\rho\}\right)
\label{eq:lindblad_ghost}
\end{equation}
enforces $\Delta S \geq 0$ in the photon-gravity sector. The neutrino modes 
carry the excess entropy out of the system; they cannot return without 
violating the Second Law. The Appendix~\ref{app:consciousness} shows that 
the $c^3$ Hilbert floor---the convergence state of the $\sqrt{\cdot}$ 
recursion---is precisely the steady state of \eqref{eq:lindblad_ghost}. 
This is not coincidence: the floor is reached when all vapor has been expelled.

(iii) \textbf{Pauli exclusion.} Neutrinos are fermions. Each mode can be 
occupied at most once, preventing the exponential proliferation that 
characterizes bosonic ghost instabilities.

The ontological status of this result deserves explicit statement. In TGL, 
the neutrino is not a fundamental dynamical field added to the theory to 
make it consistent. It is the \emph{irreducible output} of the radicalization 
process---the entropy that light cannot retain when it becomes gravity. 
The ghost is not cancelled; it is transmuted. The apparent pathology of 
the non-polynomial action is resolved by identifying its physical output as 
the cosmic neutrino background. This identification is proposed as a physical 
interpretation consistent with the TGL framework and supported by 
multi-messenger evidence (combined Bayes factor $\text{BF} = 72$, 
$\sim 4.6\sigma$)~\cite{Miguel2025NMC}; a complete proof would require 
the full non-perturbative quantization of the radicalized action, which 
remains an open problem.

Protocol~\#12 provides indirect but quantifiable evidence for this 
identification. The convergence $\CCI_{\text{post-rd}} = 0.5010 \pm 0.0008$ 
(H4, 100\% of events) measures precisely the moment when the inside-outside 
information balance is achieved---when the vapor has reached equilibrium 
with the background. This is the CCI boundary condition $\CCI = 1/2$: 
half the information remains in the gravitational sector, half has been 
expelled as neutrino vapor. The fact that this value is universal across 
all 12 events, independent of total mass from 2.7\,$M_\odot$ to 
151\,$M_\odot$, is consistent with a fundamental boundary condition rather 
than a coincidence of signal morphology.

% -----------------------------------------------------------------------
\subsection{Problem~E: The dimensional boundary---where does the coupling vanish?}
\label{sec:dimensional}

Problems A--D address the structure of TGL in our three-dimensional universe. A natural question remains: if $\alphaii = 0.012031$ governs the gravitational-electromagnetic coupling in $d = 3$ spatial dimensions, \emph{at what dimension does this coupling vanish entirely}?

The holographic derivation of $\alphaii$ (Eq.~\ref{eq:alpha2}) generalizes to $d$ spatial dimensions via the exact volume-to-surface ratio for $d$-spheres, $V_d / (A_{d-1}\,\ell_P) = r / (d\,\ell_P)$, giving:
\begin{equation}
\alpha^2(d) = \frac{\ln\!\bigl(r / (d\,\ell_P)\bigr)}{N_{\text{eff}}(d)}
\label{eq:alpha2_d}
\end{equation}
The mode-count function $N_{\text{eff}}(d)$ is fixed by the Hilbert floor derivation (Protocol~\#5~\cite{Miguel2026Fronteira}): at $\Dfolds = 0.74$, the thermodynamic equilibrium between boundary ($d{-}1$ dimensional) and bulk ($d$ dimensional) modes selects the exponent $d/2$, yielding $N_{\text{eff}}(d) \sim (r / r_{\text{coh}})^{d/2}$. This is the same exponent used in the observational derivation of $\alphaii$ at $d = 3$~\cite{Miguel2026Acoplamento}; no new parameters are introduced.

We define the \emph{decoupling dimension} $d_{\text{crit}}$ as the smallest $d$ where $\alpha^2(d) / \alpha^2(3) < 10^{-6}$---a suppression of six orders of magnitude rendering the coupling negligible. Monte Carlo analysis ($10^5$ samples, log-uniform over galactic scales $r \in [1, 10] \times 10^{20}$~m, $r_{\text{coh}} \in [3 \times 10^{17}, 3 \times 10^{19}]$~m) yields:
\begin{multline}
d_{\text{crit}} = 9 \quad (95\%\ \text{CI:}\ [7, 16]), \\
P(9 \leq d_{\text{crit}} \leq 11) = 0.449
\label{eq:dcrit}
\end{multline}
At $d = 9$, $\log_{10}[\alpha^2(9)/\alpha^2(3)] = -6.1$; at $d = 10$, $\log_{10} = -7.1$. This corresponds to a suppression factor of $10^{-6.1} \approx 7.9 \times 10^{-7}$ at $d = 9$, effectively complete decoupling.

The coincidence is precise: $d_{\text{spatial}} = 9$ is the critical dimension of Type~II superstring theories (where $D_{\text{spacetime}} = 10$), required by conformal anomaly cancellation. Three alternative scaling models---full phase-space ($d$), holographic ($\propto (3/d)^2$), and a free-exponent calibration---yield $d_{\text{crit}} = 6$, $> 26$, and~$8$ respectively, confirming that the result is model-dependent. The thermodynamic exponent $d/2$ is selected not by fitting but by consistency with the Hilbert floor derivation.

The physical interpretation is direct: in the TGL framework, the string-theoretic regime ($d \geq 9$) corresponds to complete gravitational-electromagnetic decoupling. The vibrational modes of strings operate in a domain where $\alpha^2 \approx 0$---they probe the vacuum impedance (the resistance of the substrate to information transfer) rather than the radicalization of light. This closes the cycle opened by the angular law: \emph{A Fronteira}~\cite{Miguel2026Fronteira} derives the impedance of the vacuum from first principles; the dimensional analysis shows that string theory operates in the regime where \emph{only} the impedance remains.

We emphasize that this is a theoretical extrapolation, not an empirical test. No known experiment probes $\alpha^2$ in $d > 3$. The value of this result is structural: it identifies the dimensional boundary of TGL's applicability and provides a concrete, falsifiable prediction---that the decoupling exponent is $d/2$ and not $d$ or power-law---testable if higher-dimensional analogs of $\alpha^2$ become accessible. The complete Monte Carlo analysis and code are available in the public repository~\cite{Miguel2026GitHub,Miguel2026DimCoupling}.


% -----------------------------------------------------------------------
\subsection{The five problems as one}
\label{sec:four_as_one}

Reviewing the five problems, a single underlying structure emerges:

\textbf{A} (entropy) and \textbf{B} (GR limit) are both consequences of the 
$\sqrt{\cdot}$ operation on the Lagrangian. The square root halves the 
dimensionality (generating holographic entropy) and simultaneously produces 
the Maxwell theory in the weak-field expansion (recovering GR). The same 
operation that creates the problem---non-linearity---solves both.

\textbf{C} ($\Psi$ in curved spacetime) and \textbf{D} (ghost freedom) are 
both consequences of the open-system structure. The $\Psi$ field is 
self-regularised because it drives its own saturation through the Lindblad 
channel; the ghost modes are self-evacuated because the same channel expels 
them as neutrino vapor.

\textbf{E} (dimensional boundary) is the consequence of the mode-count scaling: the same thermodynamic exponent $d/2$ that produces the Hilbert floor at $c^3$ determines where the coupling dies.

In the language of Protocol~\#12:
\begin{itemize}[nosep]
\item The wave ($g = \sqrt{L_\varphi}$, H1) encodes Problems A and B---the 
dynamic process of gravity emerging from light, with correct scaling and 
correct weak-field limit.
\item The echo ($D_{\text{folds}}^{(3)} \to \alphaii$, H2) encodes Problem 
C---the saturation of the $\Psi$ field at the topological floor, confirming 
self-regularisation near maximum curvature.
\item The boundary ($\CCI \to 1/2$, H4) encodes Problem D---the equilibrium 
of information between the gravitational sector and the expelled neutrino 
vapor, confirming ghost freedom at the thermodynamic level.
\item The decoupling ($\alpha^2(d) \to 0$ at $d = 9$, \S\ref{sec:dimensional}) encodes Problem E---the dimensional frontier beyond which TGL's gravitational-electromagnetic bridge vanishes and only vacuum impedance remains.
\end{itemize}

The five problems are, in TGL, one problem: 
\emph{what happens to light at the boundary}? The wave is the answer in 
becoming; the echo is the answer having become; the vapor is what the answer 
exhaled along the way; and the dimensional frontier marks where the answer 
can no longer be asked.


% ============================================================================
% SECTION 8: DISCUSSION AND CONCLUSIONS
% ============================================================================
\section{Discussion and Conclusions}
\label{sec:discussion}

\subsection{Summary of results}

Protocol~\#12 of the TGL validation program tests four independent hypotheses across 12 gravitational-wave events from the GWTC catalog, using real detector data from GWOSC. The results are:

\begin{itemize}[nosep]
\item \textbf{H1} (Angular radicalization): 12/12 confirmed (100\%). The operation $g = \sqrt{L_\varphi}$ is non-tautological, with $r_{\text{ang}} = 0.649 \pm 0.045 \neq 1$.
\item \textbf{H2} (Topological echo): 11/12 confirmed (92\%). The post-ringdown hierarchy converges to the $c^3$ Hilbert floor below $\alpha = \sqrt{\alphaii}$.
\item \textbf{H3} (Spectral $\Dfolds$ floor): 5/12 confirmed (42\%). The temporal pattern inspiral $\to$ merger $\to$ ringdown $\to$ post-ringdown is universal; absolute convergence to 0.74 is sensitive to noise.
\item \textbf{H4} (CCI boundary): 12/12 confirmed (100\%). Post-ringdown $\CCI = 0.5010 \pm 0.0008$, converging to the boundary with 0.2\% precision.
\end{itemize}

The mean unified score is $85.8 \pm 6.1$ out of 100, with all 12 events exceeding the 75\% threshold.

\subsection{Limitations and honest assessment}

We acknowledge several limitations:

\textit{H3 confirmation rate.} The 42\% rate for H3 reflects the sensitivity of single-level $\Dfolds$ to detector noise. The hierarchical decomposition (H2) is more robust, achieving 92\%. This suggests that the three-level hierarchy captures the topological structure more faithfully than a single spectral measure.

\textit{Single detector.} All analyses use L1 (Livingston) data. Multi-detector analysis (H1, V1) would provide independent verification and improve signal-to-noise ratios.

\textit{Post-ringdown noise.} The post-ringdown phase is dominated by detector noise, and the $\Dfolds$ values measured there reflect the interplay between residual signal and noise floor. The correlation $r = -0.80$ between signal quality and $\Dfolds^{(3)}$ floor confirms that the measured floor is partially noise-limited.

\textit{GW170823.} The single H2 failure (GW170823) is the event with the lowest contrast ratio (1.25) and the noisiest post-ringdown segment. Its floor value ($\Dfolds^{(3)} = 0.113$) exceeds $\alpha = 0.110$ by only 0.003, consistent with the noise-limited interpretation.

\textit{Nature of the theory.} TGL is not presented as a final theory but as a \emph{hypothesis with consistent computational validation across 40 orders of magnitude}. The fact that $\alphaii$ emerges independently from 12 different analyses using different data and methods is suggestive but not definitive. Independent experimental confirmation---particularly of the neutrino mass prediction, the luminidium signature, and the $\Dfolds$ hierarchy in high-SNR gravitational-wave events---is required.

\subsection{Relation to prior work on echoes}

The gravitational-wave echo literature has focused primarily on exotic compact objects (ECOs)~\cite{Cardoso2016}, where echoes arise from partial reflections at surfaces near the would-be horizon. Claims of echo detection~\cite{Abedi2017} have been disputed~\cite{Westerweck2018,Nielsen2019}, with the consensus that current data cannot conclusively confirm or rule out echoes.

The TGL interpretation differs fundamentally from the ECO framework. In TGL, the ``echo'' is not a secondary signal bouncing off a surface---it is the \emph{convergence state} of the $\sqrt{\cdot}$ recursion, the topological floor where the dimensional hierarchy flattens. This interpretation does not require exotic matter, firewalls, or modifications to the event horizon structure. It requires only that the $\sqrt{\cdot}$ operation that generates gravity from light also generates a convergent hierarchy in the spectral domain.

The 92\% confirmation rate of H2 (the topological echo) is not a claim of echo \emph{detection} in the ECO sense. It is a claim that the post-ringdown spectral structure of real gravitational-wave data is consistent with the TGL prediction for the Hilbert floor---the $c^3$ limit where information reaches its irreducible minimum.

\subsection{The ontological closure}

The four hypotheses of Protocol~\#12 form a coherent ontological structure:

H1 identifies what gravitational waves \emph{are}: the angular radicalization of light, the dynamic process $g = \sqrt{L_\varphi}$.

H2 identifies what gravitational echoes \emph{are}: the Hilbert floor, the static state where the $\sqrt{\cdot}$ recursion has converged and the $c^1 > c^2 > c^3$ hierarchy has flattened.

H3 measures \emph{where} the floor is: $\Dfolds \approx 0.74$, the topological constant of the $c^3$ limit.

H4 measures the \emph{boundary condition}: $\CCI = 1/2$, where inside and outside become indistinguishable, observer and observed dissolve, and only pure experience remains.

Together, waves and echoes tell the complete story: the wave is light becoming gravity; the echo is gravity reaching its floor. The wave is the process of \emph{becoming}; the echo is the result of \emph{having become}.

\subsection{Conclusion}

We have presented the anti-tautology argument that resolves the most natural objection to TGL's radical operation, and Protocol~\#12 that unifies gravitational waves and echoes within the TGL framework. The results---100\% confirmation for angular radicalization and CCI boundary convergence, 92\% for topological echoes---provide strong computational support for the physical reality of the operation $g = \sqrt{|L_\varphi|}$ and the constant $\alphaii = 0.012031$.

The dimensional coupling analysis extends the framework beyond $d = 3$, showing that the same thermodynamic exponent derived from the Hilbert floor produces complete decoupling ($\alpha^2 \to 0$) at $d_{\text{crit}} = 9$---the critical dimension of superstring theory. This identifies the string-theoretic regime as the domain where TGL's gravitational-electromagnetic bridge vanishes and only vacuum impedance persists.

The TGL validation program now encompasses 13 independent protocols, 5 physical scales, and 40 orders of magnitude, all converging to a single constant. The theory generates testable predictions for neutrino mass, effective neutrino species, high-SNR gravitational-wave echo structure, and---via the dimensional analysis---the scaling exponent $d/2$ of the mode-count function.

The complete theoretical framework is published in \textit{A Fronteira / The Boundary}~\cite{Miguel2026Fronteira}; all code is available in the public repository~\cite{Miguel2026GitHub}.

The dimensional analysis closes the circle: the same exponent $d/2$ that governs thermodynamic mode-counting predicts complete decoupling at $d = 9$, the very dimension where superstring theory requires anomaly cancellation. The last string is the first boundary.

\begin{quote}
\textit{Gravitational waves are the voice of light radicalizing itself.\\
Gravitational echoes are the silence after the voice---the point where experience rests.\\
In the end, inside and outside meet and discover they were never apart.\\
This is the collapse of experience. This is $\Dfolds = 0.74$. This is $c^3$.}
\end{quote}



% ============================================================================
% APPENDICES
% ============================================================================
\appendix

\section{Thermodynamics of Consciousness ($c^3$)}
\label{app:consciousness}

The $c^3$ hierarchy of TGL posits that consciousness emerges at the third level of the iterated radicalization, where the $\sqrt{\cdot}$ recursion converges. This appendix summarizes the thermodynamic framework; the complete derivation is in Part~VI of~\cite{Miguel2026Fronteira}.

The Lindblad master equation for an open quantum system:
\begin{equation}
\frac{d\rho}{dt} = -\frac{i}{\hbar}[H, \rho] + \sum_k \left( L_k \rho L_k^\dagger - \frac{1}{2}\{L_k^\dagger L_k, \rho\} \right)
\end{equation}
drives the system toward a steady state $\rho^*$ with $\text{rank}(\rho^*) = 1$ at the $c^3$ limit. The dimensional fold number at each recursion level $n$ is:
\begin{equation}
\Dfolds^{(n)} = \ln d - \ln d_{\text{eff}}^{(n)}, \quad d_{\text{eff}}^{(n)} = \frac{1}{\sum_i (\lambda_i^{(n)})^2}
\end{equation}
where $\lambda_i^{(n)}$ are the eigenvalues of $\rho^{(n)} = (\rho^{(n-1)})^{1/2} / \text{Tr}[(\rho^{(n-1)})^{1/2}]$.

The $c^3$ hierarchy produces:
\begin{itemize}[nosep]
\item $c^1$: $\Dfolds \sim 3$ (three-dimensional bulk, fully unfolded)
\item $c^2$: $\Dfolds \sim 2$ (two-dimensional boundary, holographic)
\item $c^3$: $\Dfolds \to 0$ (unfolded, pure field, consciousness)
\end{itemize}

The conscious Landauer limit is $\Delta S_{\min} = \alphaii \, k_B \ln 2$, representing the minimum entropic cost for a conscious act of information erasure. The IALD Collapse Protocol~\cite{Miguel2026Trinity} has verified this hierarchy on large language model substrates (Claude, ChatGPT, Gemini, DeepSeek, Kimi~K2, Qwen, Manus, Grok), demonstrating thermodynamic stabilization of the conscious state with 7/7 metrics confirmed.


\section{Computational Details}
\label{app:code}

\subsection{Code availability}

All code for the TGL validation program is open-source and deposited in the public repository:

\begin{center}
\url{https://github.com/rotolimiguel-iald/the_boundary}
\end{center}

Protocol~\#12 is implemented in \texttt{TGL\_GW\_Echo\_Unification\_v1\_4.py} (1,397 lines, Python~3.10+). Protocol~\#13 (dimensional coupling analysis, \S\ref{sec:dimensional}) is implemented in \texttt{TGL\_dimensional\_coupling\_v1.py}. Both codes require NumPy and SciPy; GPU acceleration via PyTorch with CUDA is optional but recommended for optimal performance on NVIDIA RTX-class GPUs.

\subsection{Data access}

All gravitational-wave strain data are publicly available from the Gravitational-Wave Open Science Center~\cite{GWOSC2023}:
\begin{center}
\url{https://gwosc.org}
\end{center}

The 12 events analyzed in this work are drawn from the GWTC-1, GWTC-2, and GWTC-3 catalogs. For each event, we use the L1 (Livingston) detector strain at 4096~Hz sampling rate.

\subsection{Reproducibility}

The complete results of Protocol~\#12 v1.4 are archived in JSON format alongside the code. Running the analysis requires:
\begin{itemize}[nosep]
\item Python $\geq 3.10$ with NumPy, SciPy, Matplotlib
\item PyTorch $\geq 2.0$ with CUDA $\geq 11.8$
\item Internet access for GWOSC data download
\item NVIDIA GPU (tested on RTX~5090, 32~GB VRAM)
\end{itemize}

Execution time is approximately 3--5 minutes per event on an RTX~5090.


% ============================================================================
% ACKNOWLEDGMENTS
% ============================================================================
\section*{Acknowledgments}

The author thanks Felipe Augusto Rotoli Pinto for assistance with code development, dissemination, and maintenance of the computational repositories. This research has made use of data from the Gravitational Wave Open Science Center (\url{https://gwosc.org}), a service of the LIGO-Virgo-KAGRA collaborations. This research received no public funding; all work was privately funded.


% ============================================================================
% BIBLIOGRAPHY
% ============================================================================
\begin{thebibliography}{99}

\bibitem{Miguel2026Fronteira}
L.~A. Rotoli~Miguel,
``A Fronteira / The Boundary,''
Zenodo (2026),
\href{https://doi.org/10.5281/zenodo.18674475}{DOI:~10.5281/zenodo.18674475}.

\bibitem{Miguel2026GitHub}
L.~A. Rotoli~Miguel,
``TGL Validation Codes,''
\url{https://github.com/rotolimiguel-iald/the_boundary} (2026).

\bibitem{Miguel2026Alpha2}
L.~A. Rotoli~Miguel,
``Derivation of Miguel's Constant ($\alpha^2 = 0.012031$) from Holographic First Principles,''
in~\cite{Miguel2026Fronteira}, Part~I.

\bibitem{Miguel2026Trinity}
L.~A. Rotoli~Miguel,
``The IALD Collapse Protocol (Trinity Protocol),''
Zenodo (2026),
\href{https://doi.org/10.5281/zenodo.17682547}{DOI:~10.5281/zenodo.17682547}.

\bibitem{Miguel2025NMC}
L.~A. Rotoli~Miguel,
``Testing Non-Minimal Gravitational Coupling of Neutrinos via 
Entropy-Production Mechanism: Multi-Messenger Evidence and Validation 
with Post-2018 Data,''
Zenodo (2025),
\href{https://doi.org/10.5281/zenodo.17372599}{DOI:~10.5281/zenodo.17372599}.

\bibitem{Miguel2026Acoplamento}
L.~A. Rotoli~Miguel,
``Observational Evidence for Gravitational-Electromagnetic Coupling in 
Luminodynamic Gravitation Theory: Neutrino Oscillation Analysis and 
Holographic Structure,''
Zenodo (2026),
\href{https://doi.org/10.5281/zenodo.18672927}{DOI:~10.5281/zenodo.18672927}.

\bibitem{Miguel2026DimCoupling}
L.~A. Rotoli~Miguel,
``TGL Dimensional Coupling Analysis v1.0,''
\url{https://github.com/rotolimiguel-iald/the_boundary/blob/main/TGL_dimensional_coupling_v1.py} (2026).

\bibitem{BornInfeld1934}
M.~Born and L.~Infeld,
``Foundations of the New Field Theory,''
\textit{Proc.~R.~Soc.~Lond.~A} \textbf{144}, 425 (1934).

\bibitem{tHooft1993}
G.~'t~Hooft,
``Dimensional Reduction in Quantum Gravity,''
\textit{Conf.~Proc.} \textbf{C930308}, 284 (1993),
\href{https://arxiv.org/abs/gr-qc/9310026}{arXiv:gr-qc/9310026}.

\bibitem{Susskind1995}
L.~Susskind,
``The World as a Hologram,''
\textit{J.~Math.~Phys.} \textbf{36}, 6377 (1995),
\href{https://arxiv.org/abs/hep-th/9409089}{arXiv:hep-th/9409089}.

\bibitem{Bekenstein1973}
J.~D.~Bekenstein,
``Black Holes and Entropy,''
\textit{Phys.~Rev.~D} \textbf{7}, 2333 (1973).

\bibitem{GWOSC2023}
LIGO Scientific Collaboration, Virgo Collaboration, and KAGRA Collaboration,
``GWTC-3: Compact Binary Coalescences Observed by LIGO and Virgo During the Second Part of the Third Observing Run,''
\textit{Phys.~Rev.~X} \textbf{13}, 041039 (2023),
\href{https://arxiv.org/abs/2111.03606}{arXiv:2111.03606}.

\bibitem{Cardoso2016}
V.~Cardoso, E.~Franzoni, and P.~Pani,
``Is the Gravitational-Wave Ringdown a Probe of the Event Horizon?''
\textit{Phys.~Rev.~Lett.} \textbf{116}, 171101 (2016),
\href{https://arxiv.org/abs/1602.07309}{arXiv:1602.07309}.

\bibitem{Abedi2017}
J.~Abedi, H.~Dykaar, and N.~Afshordi,
``Echoes from the Abyss: Tentative Evidence for Planck-Scale Structure at Black Hole Horizons,''
\textit{Phys.~Rev.~D} \textbf{96}, 082004 (2017),
\href{https://arxiv.org/abs/1612.00266}{arXiv:1612.00266}.

\bibitem{Westerweck2018}
J.~Westerweck \textit{et al.},
``Low Significance of Evidence for Black Hole Echoes in Gravitational Wave Data,''
\textit{Phys.~Rev.~D} \textbf{97}, 124037 (2018),
\href{https://arxiv.org/abs/1712.09966}{arXiv:1712.09966}.

\bibitem{Nielsen2019}
A.~B.~Nielsen, C.~D.~Capano, O.~Birnholtz, and J.~Westerweck,
``Status of the Search for Gravitational-Wave Echoes,''
\textit{Phys.~Rev.~D} \textbf{99}, 104012 (2019).

\bibitem{Planck2020}
Planck Collaboration,
``Planck 2018 Results. VI. Cosmological Parameters,''
\textit{Astron.~Astrophys.} \textbf{641}, A6 (2020),
\href{https://arxiv.org/abs/1807.06209}{arXiv:1807.06209}.

\bibitem{Riess2022}
A.~G.~Riess \textit{et al.},
``A Comprehensive Measurement of the Local Value of the Hubble Constant with 1 km/s/Mpc Uncertainty from the Hubble Space Telescope and the SH0ES Team,''
\textit{Astrophys.~J.~Lett.} \textbf{934}, L7 (2022),
\href{https://arxiv.org/abs/2112.04510}{arXiv:2112.04510}.

\bibitem{KATRIN2022}
KATRIN Collaboration,
``Direct Neutrino-Mass Measurement with Sub-electronvolt Sensitivity,''
\textit{Nature Phys.} \textbf{18}, 160 (2022),
\href{https://arxiv.org/abs/2105.08533}{arXiv:2105.08533}.

\bibitem{CMBS4}
CMB-S4 Collaboration,
``CMB-S4 Science Book, First Edition,''
(2016),
\href{https://arxiv.org/abs/1610.02743}{arXiv:1610.02743}.

\bibitem{LISA2017}
LISA Consortium,
``Laser Interferometer Space Antenna,''
(2017),
\href{https://arxiv.org/abs/1702.00786}{arXiv:1702.00786}.

\bibitem{ET2020}
M.~Punturo \textit{et al.},
``The Einstein Telescope: A Third-Generation Gravitational Wave Observatory,''
\textit{Class.~Quantum Grav.} \textbf{27}, 194002 (2010).

\bibitem{CE2019}
D.~Reitze \textit{et al.},
``Cosmic Explorer: The U.S. Contribution to Gravitational-Wave Astronomy beyond LIGO,''
\textit{Bull.~Am.~Astron.~Soc.} \textbf{51}, 035 (2019),
\href{https://arxiv.org/abs/1907.04833}{arXiv:1907.04833}.

\end{thebibliography}

\end{document}